% fleqn aligns equations to the left, a4 paper size, 11pt font, article class
\documentclass[fleqn,a4paper,11pt]{article}
\title{Polygonal Cannonball Numbers}
\author{Izaak van Dongen}

\usepackage{mystyle}

\begin{document}
\maketitle\thispagestyle{empty} % no page number under title
\tableofcontents

\section{Introduction}

Recently I watched the video \url{https://www.youtube.com/watch?v=q6L06pyt9CA},
featuring Matt Parker.  Being a huge fan of Matt and of Numperphile, and being
rather full of myself, my first response was naturally one of doubt (also
because of the spirit of mathematical enquiry and all that). I decided to have
my own crack at the problem, since I my rough estimation for the complexity of
this task did not seem to match with the claim that it could take a whole night
to compute all cannonball numbers under \(10^9\), for \(s \lesssim 31265\).

I reasoned that it should be a roughly \(\BigO(1)\) operation as we can find
the \(n\)th term of the base-\(s\) polygonal numbers \(P(s, n)\), which will
be quadratic in \(n\), and solve it for \(n\) with the quadratic formula, so
to check if some cannonball numbers \(C(s, n_c)\) is polygonal we just see
if the corresponding \(n_p\) is an integer. Now \(10^9\) is a fairly small
number. Seeing as my CPU's clockspeed is in the range of gigahertz, and
we're just checking a tiny fraction of those numbers as we're just computing
the cannonball numbers under this limit, it seems reasonable that this
should be doable fairly fast.

I've thought about the problem of higher-dimensional stacks of cannonballs (ie
the ones formed by adding up the cannonball numbers), but I've not done anything
about it.

\section{The Maths}

Indeed, this approach does seem to work. Almost by definition we have the
recurrence in polygonal numbers
\begin{equation*}
P(s, n) = P(s, n - 1) + n(s - 2) - (s - 3)
\end{equation*}
so we can use
\begin{align*}
P(s, n) &= \sum_{r = 1}^n P(s, r) - P(s, r - 1) \\
    &= \sum_{r = 1}^n (n(s - 2) - (s - 3)) \\
    &= \frac 12 n(n + 1)(s - 2) - n(s - 3) \\
    &= \frac{n^2(s - 2) - n(s - 4)} 2
\end{align*}
Fortunately this seems to agree with what Wikipedia thinks. Now, we have
\begin{alignat*}{2}
&& 0 &= (s - 2)n^2 - (s - 4)n - 2P(s, n) \\
&\implies& n &= \frac{s - 4 + \sqrt{(s - 4)^2 + 8(s - 2)P(s, n)}}{2s - 4}
\end{alignat*}
Wikipedia still seems to think we're on track.

Another result that I don't really use is that
\begin{align*}
C(s, n) &= \sum_{r = 1}^n P(s, n) \\
    &= \frac 12 \sum_{r = 1}^n (n^2(s - 2) - n(s - 4)) \\
    &= \frac 12 \pqty{\frac{n(n + 1)(2n + 1)(s - 2)} 6
                    - \frac{n(n + 1)(s - 4)} 2} \\
    &= \frac 1{12}n(n + 1)\bqty{(2n + 1)(s - 2) - 3(s - 4)}
\end{align*}
In fact I've only used this in verification of the results.

Regardless, now we need only work our way up the \(C(s, n)\)s using the
recurrence \(C(s, n) = P(s, n) + C(s, n - 1)\), and check for each if the
quadratic formula gives an integer result. This is most easily done by checking
if the discriminant is a perfect square and then checking that the denominator
divides the numerator.

\section{The Programming}

For speeeeeeed I implemented this in C (although there is a long abandoned
parallel Python implementation). I used 128-bit integers to be on the safe side,
as \(10^{19}\) is a little small for my liking. This meant I had to do a lot of
messing around to get things to actually display in base 10.

I did briefly consider either implementing or importing some kind of arbitrary
precision integer arithmetic functionality, but then I decided I wasn't going to
run it on anything fast enough to have to worry about that, and I have better
things to do.

There's also a slick little progress update that gets printed to STDERR, and a
number of zsh scripts to save me typing.

\section{The Ugly}

Table \ref{tab_ugly} lists all the solutions that I've found, so far. The \TeX
source of the table is in \texttt{../src/tab.tex}, which is derived from
\texttt{../src/c/solutions/*}.

\begin{longtable}{*4r}
\toprule
\boldmath \(s\) & \boldmath \(C(s, n_c) = P(s, n_p)\)
& \boldmath \(n_p\) & \boldmath \(n_c\) \\
\midrule
3 & 10 & 4 & 3 \\
3 & 120 & 15 & 8 \\
3 & 1540 & 55 & 20 \\
3 & 7140 & 119 & 34 \\
4 & 4900 & 70 & 24 \\
6 & 946 & 22 & 11 \\
8 & 1045 & 19 & 10 \\
8 & 5985 & 45 & 18 \\
8 & 123395663059845 & 6413415 & 49785 \\
8 & 774611255177760 & 16068720 & 91839 \\
10 & 175 & 7 & 5 \\
10 & 368050005576 & 303336 & 6511 \\
11 & 23725 & 73 & 25 \\
11 & 1519937678700 & 581175 & 10044 \\
11 & 7248070597636 & 1269127 & 16906 \\
14 & 441 & 9 & 6 \\
14 & 195661 & 181 & 46 \\
17 & 975061 & 361 & 73 \\
17 & 1580765544996 & 459096 & 8583 \\
20 & 3578401 & 631 & 106 \\
23 & 10680265 & 1009 & 145 \\
26 & 27453385 & 1513 & 190 \\
29 & 63016921 & 2161 & 241 \\
30 & 23001 & 41 & 17 \\
32 & 132361021 & 2971 & 298 \\
35 & 258815701 & 3961 & 361 \\
38 & 477132085 & 5149 & 430 \\
41 & 55202400 & 1683 & 204 \\
41 & 837244045 & 6553 & 505 \\
43 & 245905 & 110 & 33 \\
44 & 1408778281 & 8191 & 586 \\
47 & 2286380881 & 10081 & 673 \\
50 & 314755 & 115 & 34 \\
50 & 3595928401 & 12241 & 766 \\
53 & 5501691505 & 14689 & 865 \\
56 & 8214519205 & 17443 & 970 \\
59 & 12001111741 & 20521 & 1081 \\
60 & 1785508245600 & 248132 & 5695 \\
62 & 17194450141 & 23941 & 1198 \\
65 & 24205450501 & 27721 & 1321 \\
68 & 33535911025 & 31879 & 1450 \\
71 & 45792819865 & 36433 & 1585 \\
74 & 61704091801 & 41401 & 1726 \\
77 & 82135801801 & 46801 & 1873 \\
80 & 108110983501 & 52651 & 2026 \\
83 & 140830060645 & 58969 & 2185 \\
86 & 181692979525 & 65773 & 2350 \\
88 & 48280 & 34 & 15 \\
89 & 232323110461 & 73081 & 2521 \\
92 & 294592986361 & 80911 & 2698 \\
95 & 370651946401 & 89281 & 2881 \\
98 & 462955752865 & 98209 & 3070 \\
101 & 574298249185 & 107713 & 3265 \\
104 & 707845127221 & 117811 & 3466 \\
107 & 867169871821 & 128521 & 3673 \\
110 & 1056291950701 & 139861 & 3886 \\
113 & 1279717317685 & 151849 & 4105 \\
116 & 1542481297345 & 164503 & 4330 \\
119 & 1850193919081 & 177841 & 4561 \\
122 & 2209087768681 & 191881 & 4798 \\
125 & 2626068425401 & 206641 & 5041 \\
128 & 3108767552605 & 222139 & 5290 \\
131 & 3665598710005 & 238393 & 5545 \\
134 & 4305815955541 & 255421 & 5806 \\
137 & 5039575304941 & 273241 & 6073 \\
140 & 5877999117001 & 291871 & 6346 \\
143 & 6833243472625 & 311329 & 6625 \\
145 & 101337426 & 1191 & 162 \\
146 & 7918568615665 & 331633 & 6910 \\
149 & 9148412523601 & 352801 & 7201 \\
152 & 10538467676101 & 374851 & 7498 \\
155 & 12105761089501 & 397801 & 7801 \\
158 & 13868737685245 & 421669 & 8110 \\
161 & 15847347060325 & 446473 & 8425 \\
164 & 18063133727761 & 472231 & 8746 \\
167 & 20539330895161 & 498961 & 9073 \\
170 & 23300957849401 & 526681 & 9406 \\
173 & 26374921015465 & 555409 & 9745 \\
176 & 29790118757485 & 585163 & 10090 \\
179 & 33577549990021 & 615961 & 10441 \\
182 & 37770426667621 & 647821 & 10798 \\
185 & 42404290220701 & 680761 & 11161 \\
188 & 47517132005785 & 714799 & 11530 \\
191 & 53149517838145 & 749953 & 11905 \\
194 & 59344716674881 & 786241 & 12286 \\
197 & 66148833516481 & 823681 & 12673 \\
200 & 73610946594901 & 862291 & 13066 \\
203 & 81783248916205 & 902089 & 13465 \\
206 & 90721194225805 & 943093 & 13870 \\
209 & 100483647464341 & 985321 & 14281 \\
212 & 111133039782241 & 1028791 & 14698 \\
215 & 122735528181001 & 1073521 & 15121 \\
218 & 135361159849225 & 1119529 & 15550 \\
221 & 149084041261465 & 1166833 & 15985 \\
224 & 163982512107901 & 1215451 & 16426 \\
227 & 180139324122901 & 1265401 & 16873 \\
230 & 197641824880501 & 1316701 & 17326 \\
233 & 216582146624845 & 1369369 & 17785 \\
236 & 237057400203625 & 1423423 & 18250 \\
239 & 259169874172561 & 1478881 & 18721 \\
242 & 283027239138961 & 1535761 & 19198 \\
245 & 308742757412401 & 1594081 & 19681 \\
248 & 336435498030565 & 1653859 & 20170 \\
251 & 366230557228285 & 1715113 & 20665 \\
254 & 398259284417821 & 1777861 & 21166 \\
257 & 432659513748421 & 1842121 & 21673 \\
260 & 469575801313201 & 1907911 & 22186 \\
263 & 509159668071385 & 1975249 & 22705 \\
266 & 551569848553945 & 2044153 & 23230 \\
269 & 596972545420681 & 2114641 & 23761 \\
272 & 645541689936781 & 2186731 & 24298 \\
275 & 697459208436901 & 2260441 & 24841 \\
276 & 801801 & 77 & 26 \\
278 & 752915294844805 & 2335789 & 25390 \\
281 & 812108689316605 & 2412793 & 25945 \\
284 & 875246963075641 & 2491471 & 26506 \\
287 & 942546809507041 & 2571841 & 27073 \\
290 & 1014234341580001 & 2653921 & 27646 \\
293 & 1090545395665825 & 2737729 & 28225 \\
296 & 1171725841819765 & 2823283 & 28810 \\
299 & 1258031900594701 & 2910601 & 29401 \\
302 & 1349730466454701 & 2999701 & 29998 \\
305 & 1447099437856501 & 3090601 & 30601 \\
308 & 1550428054066945 & 3183319 & 31210 \\
311 & 1660017238784425 & 3277873 & 31825 \\
314 & 1776179950632361 & 3374281 & 32446 \\
317 & 1899241540592761 & 3472561 & 33073 \\
320 & 2029540116447901 & 3572731 & 33706 \\
322 & 1169686 & 86 & 28 \\
323 & 2167426914298165 & 3674809 & 34345 \\
326 & 2313266677224085 & 3778813 & 34990 \\
329 & 2467438041160621 & 3884761 & 35641 \\
332 & 2630333928051721 & 3992671 & 36298 \\
335 & 2802361946353201 & 4102561 & 36961 \\
338 & 2983944798951985 & 4214449 & 37630 \\
341 & 3175520698569745 & 4328353 & 38305 \\
344 & 3377543790718981 & 4444291 & 38986 \\
347 & 3590484584279581 & 4562281 & 39673 \\
350 & 3814830389763901 & 4682341 & 40366 \\
353 & 4051085765338405 & 4804489 & 41065 \\
356 & 4299772970669905 & 4928743 & 41770 \\
359 & 4561432428664441 & 5055121 & 42481 \\
362 & 4836623195166841 & 5183641 & 43198 \\
365 & 5125923436689001 & 5314321 & 43921 \\
368 & 5429930916234925 & 5447179 & 44650 \\
371 & 5749263487290565 & 5582233 & 45385 \\
374 & 15064335000 & 9000 & 624 \\
374 & 6084559596046501 & 5719501 & 46126 \\
377 & 6436478791921501 & 5859001 & 46873 \\
380 & 6805702246455001 & 6000751 & 47626 \\
383 & 7192933280636545 & 6144769 & 48385 \\
386 & 7598897900740225 & 6291073 & 49150 \\
389 & 8024345342732161 & 6439681 & 49921 \\
392 & 8470048625319061 & 6590611 & 50698 \\
395 & 8936805111705901 & 6743881 & 51481 \\
398 & 9425437080130765 & 6899509 & 52270 \\
401 & 9936792303244885 & 7057513 & 53065 \\
404 & 10471744636405921 & 7217911 & 53866 \\
407 & 11031194614952521 & 7380721 & 54673 \\
410 & 11616070060528201 & 7545961 & 55486 \\
413 & 12227326696522585 & 7713649 & 56305 \\
416 & 12865948772698045 & 7883803 & 57130 \\
419 & 13532949699069781 & 8056441 & 57961 \\
422 & 14229372689107381 & 8231581 & 58798 \\
425 & 14956291412325901 & 8409241 & 59641 \\
428 & 15714810656334505 & 8589439 & 60490 \\
431 & 16506066998410705 & 8772193 & 61345 \\
434 & 17331229486668241 & 8957521 & 62206 \\
437 & 18191500330886641 & 9145441 & 63073 \\
440 & 19088115603070501 & 9335971 & 63946 \\
443 & 20022345947806525 & 9529129 & 64825 \\
446 & 20995497302486365 & 9724933 & 65710 \\
449 & 22008911627463301 & 9923401 & 66601 \\
452 & 23063967646210801 & 10124551 & 67498 \\
455 & 24162081595551001 & 10328401 & 68401 \\
458 & 25304707986021145 & 10534969 & 69310 \\
461 & 26493340372446025 & 10744273 & 70225 \\
464 & 27729512134784461 & 10956331 & 71146 \\
467 & 29014797269317861 & 11171161 & 72073 \\
470 & 30350811190248901 & 11388781 & 73006 \\
473 & 31739211541778365 & 11609209 & 73945 \\
476 & 33181699020728185 & 11832463 & 74890 \\
479 & 34680018209778721 & 12058561 & 75841 \\
482 & 36235958421388321 & 12287521 & 76798 \\
485 & 37851354552463201 & 12519361 & 77761 \\
488 & 39528087949845685 & 12754099 & 78730 \\
491 & 41268087286688845 & 12991753 & 79705 \\
494 & 43073329449785581 & 13232341 & 80686 \\
497 & 44945840437920181 & 13475881 & 81673 \\
500 & 46887696271310401 & 13722391 & 82666 \\
503 & 48901023912208105 & 13971889 & 83665 \\
506 & 50988002196726505 & 14224393 & 84670 \\
509 & 53150862777962041 & 14479921 & 85681 \\
512 & 55391891080478941 & 14738491 & 86698 \\
515 & 57713427266224501 & 15000121 & 87721 \\
518 & 60117867211943125 & 15264829 & 88750 \\
521 & 62607663498157165 & 15532633 & 89785 \\
524 & 65185326409782601 & 15803551 & 90826 \\
527 & 67853424948447601 & 16077601 & 91873 \\
530 & 70614587856582001 & 16354801 & 92926 \\
533 & 73471504653345745 & 16635169 & 93985 \\
536 & 76426926682464325 & 16918723 & 95050 \\
539 & 79483668172039261 & 17205481 & 96121 \\
542 & 82644607306401661 & 17495461 & 97198 \\
545 & 85912687310076901 & 17788681 & 98281 \\
548 & 89290917543928465 & 18085159 & 99370 \\
551 & 92782374613548985 & 18384913 & 100465 \\
554 & 96390203489966521 & 18687961 & 101566 \\
557 & 100117618642734121 & 18994321 & 102673 \\
560 & 103967905185470701 & 19304011 & 103786 \\
563 & 107944420033921285 & 19617049 & 104905 \\
566 & 112050593076604645 & 19933453 & 106030 \\
569 & 116289928358116381 & 20253241 & 107161 \\
572 & 120666005275155481 & 20576431 & 108298 \\
575 & 125182479785342401 & 20903041 & 109441 \\
578 & 129843085628896705 & 21233089 & 110590 \\
581 & 134651635563242305 & 21566593 & 111745 \\
584 & 139612022610608341 & 21903571 & 112906 \\
587 & 144728221318693741 & 22244041 & 114073 \\
590 & 150004289034463501 & 22588021 & 115246 \\
593 & 155444367191144725 & 22935529 & 116425 \\
596 & 161052682608490465 & 23286583 & 117610 \\
599 & 166833548806379401 & 23641201 & 118801 \\
602 & 172791367331819401 & 23999401 & 119998 \\
605 & 178930629099423001 & 24361201 & 121201 \\
608 & 185255915745422845 & 24726619 & 122410 \\
611 & 191771900995295125 & 25095673 & 123625 \\
614 & 198483352045059061 & 25468381 & 124846 \\
617 & 205395130956320461 & 25844761 & 126073 \\
620 & 212512196065127401 & 26224831 & 127306 \\
623 & 219839603404706065 & 26608609 & 128545 \\
626 & 227382508142144785 & 26996113 & 129790 \\
629 & 235146166029094321 & 27387361 & 131041 \\
632 & 243135934866552421 & 27782371 & 132298 \\
635 & 251357275983800701 & 28181161 & 133561 \\
638 & 259815755731561885 & 28583749 & 134830 \\
641 & 268517046989445445 & 28990153 & 136105 \\
644 & 277466930687749681 & 29400391 & 137386 \\
647 & 286671297343688281 & 29814481 & 138673 \\
650 & 296136148612109401 & 30232441 & 139966 \\
653 & 305867598850775305 & 30654289 & 141265 \\
656 & 315871876700270605 & 31080043 & 142570 \\
659 & 326155326678607141 & 31509721 & 143881 \\
662 & 336724410790593541 & 31943341 & 145198 \\
665 & 347585710152037501 & 32380921 & 146521 \\
668 & 358745926628848825 & 32822479 & 147850 \\
671 & 370211884491111265 & 33268033 & 149185 \\
674 & 381990532082191201 & 33717601 & 150526 \\
677 & 394088943502951201 & 34171201 & 151873 \\
680 & 406514320311136501 & 34628851 & 153226 \\
683 & 419273993236002445 & 35090569 & 154585 \\
686 & 432375423908250925 & 35556373 & 155950 \\
689 & 445826206605343861 & 36026281 & 157321 \\
692 & 459634070012261761 & 36500311 & 158698 \\
695 & 473806878997775401 & 36978481 & 160081 \\
698 & 488352636406298665 & 37460809 & 161470 \\
701 & 503279484865390585 & 37947313 & 162865 \\
704 & 518595708608974621 & 38438011 & 164266 \\
707 & 534309735316343221 & 38932921 & 165673 \\
710 & 550430137967015701 & 39432061 & 167086 \\
713 & 566965636711517485 & 39935449 & 168505 \\
716 & 583925100758148745 & 40443103 & 169930 \\
719 & 601317550275810481 & 40955041 & 171361 \\
722 & 619152158312956081 & 41471281 & 172798 \\
725 & 637438252732736401 & 41991841 & 174241 \\
728 & 656185318164406405 & 42516739 & 175690 \\
731 & 675402997971061405 & 43045993 & 177145 \\
734 & 695101096233770941 & 43579621 & 178606 \\
737 & 715289579752178341 & 44117641 & 180073 \\
740 & 735978580061634001 & 44660071 & 181546 \\
743 & 757178395466930425 & 45206929 & 183025 \\
746 & 778899493092707065 & 45758233 & 184510 \\
749 & 801152510950593001 & 46314001 & 186001 \\
752 & 823948260023155501 & 46874251 & 187498 \\
755 & 847297726364722501 & 47439001 & 189001 \\
758 & 871212073219147045 & 48008269 & 190510 \\
761 & 895702643154581725 & 48582073 & 192025 \\
764 & 920780960215331161 & 49160431 & 193546 \\
767 & 946458732090850561 & 49743361 & 195073 \\
770 & 972747852301958401 & 50330881 & 196606 \\
773 & 999660402404331265 & 50923009 & 198145 \\
776 & 1027208654209348885 & 51519763 & 199690 \\
779 & 1055405072022357421 & 52121161 & 201241 \\
782 & 1084262314898419021 & 52727221 & 202798 \\
785 & 1113793238915615701 & 53337961 & 204361 \\
788 & 1144010899465975585 & 53953399 & 205930 \\
791 & 1174928553564089545 & 54573553 & 207505 \\
794 & 1206559662173486281 & 55198441 & 209086 \\
797 & 1238917892550833881 & 55828081 & 210673 \\
800 & 1272017120608035901 & 56462491 & 212266 \\
803 & 1305871433292290005 & 57101689 & 213865 \\
806 & 1340495130984177205 & 57745693 & 215470 \\
809 & 1375902729913849741 & 58394521 & 217081 \\
812 & 1412108964595385641 & 59048191 & 218698 \\
815 & 1449128790279378001 & 59706721 & 220321 \\
818 & 1486977385423827025 & 60370129 & 221950 \\
821 & 1525670154183402865 & 61038433 & 223585 \\
823 & 197427385 & 694 & 113 \\
824 & 1565222728917147301 & 61711651 & 225226 \\
827 & 1605650972714682301 & 62389801 & 226873 \\
830 & 1646970981940993501 & 63072901 & 228526 \\
833 & 1689199088799856645 & 63760969 & 230185 \\
836 & 1732351863915975025 & 64454023 & 231850 \\
839 & 1776446118935895961 & 65152081 & 233521 \\
842 & 1821498909147774361 & 65855161 & 235198 \\
845 & 1867527536120051401 & 66563281 & 236881 \\
848 & 1914549550359116365 & 67276459 & 238570 \\
851 & 1962582753986019685 & 67994713 & 240265 \\
854 & 2011645203432305221 & 68718061 & 241966 \\
857 & 2061755212155029821 & 69446521 & 243673 \\
860 & 2112931353371038201 & 70180111 & 245386 \\
863 & 2165192462810561185 & 70918849 & 247105 \\
866 & 2218557641490205345 & 71662753 & 248830 \\
869 & 2273046258505402081 & 72411841 & 250561 \\
872 & 2328677953842384181 & 73166131 & 252298 \\
875 & 2385472641209757901 & 73925641 & 254041 \\
878 & 2443450510889738605 & 74690389 & 255790 \\
881 & 2502632032609118005 & 75460393 & 257545 \\
884 & 2563037958430031041 & 76235671 & 259306 \\
887 & 2624689325660590441 & 77016241 & 261073 \\
890 & 2687607459785457001 & 77802121 & 262846 \\
893 & 2751813977416413625 & 78593329 & 264625 \\
896 & 2817330789263011165 & 79389883 & 266410 \\
899 & 2884180103123354101 & 80191801 & 268201 \\
902 & 2952384426895094101 & 80999101 & 269998 \\
905 & 3021966571606699501 & 81811801 & 271801 \\
908 & 3092949654469068745 & 82629919 & 273610 \\
911 & 3165357101947555825 & 83453473 & 275425 \\
914 & 3239212652854475761 & 84282481 & 277246 \\
917 & 3314540361462158161 & 85116961 & 279073 \\
920 & 3391364600636616901 & 85956931 & 280906 \\
923 & 3469710064991903965 & 86802409 & 282745 \\
926 & 3549601774065215485 & 87653413 & 284590 \\
929 & 3631065075512818021 & 88509961 & 286441 \\
932 & 3714125648326863121 & 89372071 & 288298 \\
935 & 3798809506073158201 & 90239761 & 290161 \\
938 & 3885143000149961785 & 91113049 & 292030 \\
941 & 3973152823067871145 & 91991953 & 293905 \\
944 & 4062866011750870381 & 92876491 & 295786 \\
947 & 4154309950858606981 & 93766681 & 297673 \\
950 & 4247512376129964901 & 94662541 & 299566 \\
953 & 4342501377748002205 & 95564089 & 301465 \\
956 & 4439305403726321305 & 96471343 & 303370 \\
959 & 4537953263316939841 & 97384321 & 305281 \\
962 & 4638474130439730241 & 98303041 & 307198 \\
965 & 4740897547133496001 & 99227521 & 309121 \\
968 & 4845253427028752725 & 100157779 & 311050 \\
971 & 4951572058842281965 & 101093833 & 312985 \\
974 & 5059884109893525901 & 102035701 & 314926 \\
977 & 5170220629642890901 & 102983401 & 316873 \\
980 & 5282613053252028001 & 103936951 & 318826 \\
983 & 5397093205166158345 & 104896369 & 320785 \\
986 & 5513693302718511625 & 105861673 & 322750 \\
989 & 5632445959756945561 & 106832881 & 324721 \\
992 & 5753384190292814461 & 107810011 & 326698 \\
995 & 5876541412172154901 & 108793081 & 328681 \\
998 & 6001951450769256565 & 109782109 & 330670 \\
1001 & 6129648542702686285 & 110777113 & 332665 \\
1004 & 6259667339573833321 & 111778111 & 334666 \\
1007 & 6392042911728043921 & 112785121 & 336673 \\
1010 & 6526810752038413201 & 113798161 & 338686 \\
1013 & 6664006779712302385 & 114817249 & 340705 \\
1016 & 6803667344120649445 & 115842403 & 342730 \\
1019 & 6945829228650141181 & 116873641 & 344761 \\
1022 & 7090529654578314781 & 117910981 & 346798 \\
1025 & 7237806284971656901 & 118954441 & 348841 \\
1028 & 7387697228606768305 & 120004039 & 350890 \\
1031 & 7540241043914662105 & 121059793 & 352945 \\
1034 & 7695476742948263641 & 122121721 & 355006 \\
1037 & 7853443795373180041 & 123189841 & 357073 \\
1040 & 8014182132481807501 & 124264171 & 359146 \\
1043 & 8177732151230844325 & 125344729 & 361225 \\
1046 & 8344134718302277765 & 126431533 & 363310 \\
1049 & 8513431174187912701 & 127524601 & 365401 \\
1052 & 8685663337297510201 & 128623951 & 367498 \\
1055 & 8860873508090604001 & 129729601 & 369601 \\
1058 & 9039104473232062945 & 130841569 & 371710 \\
1061 & 9220399509771467425 & 131959873 & 373825 \\
1064 & 9404802389346367861 & 133084531 & 375946 \\
1067 & 9592357382409493261 & 134215561 & 378073 \\
1070 & 9783109262479977901 & 135352981 & 380206 \\
1073 & 9977103310418674165 & 136496809 & 382345 \\
1076 & 10174385318727619585 & 137647063 & 384490 \\
1079 & 10375001595873726121 & 138803761 & 386641 \\
1082 & 10578998970636759721 & 139966921 & 388798 \\
1085 & 10786424796481678201 & 141136561 & 390961 \\
1088 & 10997326955955395485 & 142312699 & 393130 \\
1091 & 11211753865108040245 & 143495353 & 395305 \\
1094 & 11429754477938776981 & 144684541 & 397486 \\
1097 & 11651378290866257581 & 145880281 & 399673 \\
1100 & 11876675347223771401 & 147082591 & 401866 \\
1103 & 12105696241779161905 & 148291489 & 404065 \\
1106 & 12338492125279577905 & 149506993 & 406270 \\
1109 & 12575114709021127441 & 150729121 & 408481 \\
1112 & 12815616269443502341 & 151957891 & 410698 \\
1115 & 13060049652749641501 & 153193321 & 412921 \\
1118 & 13308468279550500925 & 154435429 & 415150 \\
1121 & 13560926149534998565 & 155684233 & 417385 \\
1124 & 13817477846165202001 & 156939751 & 419626 \\
1127 & 14078178541396827001 & 158202001 & 421873 \\
1130 & 14343084000425115001 & 159471001 & 424126 \\
1133 & 14612250586456157545 & 160746769 & 426385 \\
1136 & 14885735265503735725 & 162029323 & 428650 \\
1139 & 15163595611211742661 & 163318681 & 430921 \\
1142 & 15445889809702257061 & 164614861 & 433198 \\
1145 & 15732676664449335901 & 165917881 & 435481 \\
1148 & 16024015601178594265 & 167227759 & 437770 \\
1151 & 16319966672792640385 & 168544513 & 440065 \\
1152 & 149979784926720 & 510720 & 9215 \\
1154 & 16620590564322433921 & 169868161 & 442366 \\
1157 & 16925948597904635521 & 171198721 & 444673 \\
1160 & 17236102737785015701 & 172536211 & 446986 \\
1163 & 17551115595347991085 & 173880649 & 449305 \\
1166 & 17871050434172356045 & 175232053 & 451630 \\
1169 & 18195971175113277781 & 176590441 & 453961 \\
1172 & 18525942401410622881 & 177955831 & 456298 \\
1175 & 18861029363823683401 & 179328241 & 458641 \\
1178 & 19201297985792370505 & 180707689 & 460990 \\
1181 & 19546814868624943705 & 182094193 & 463345 \\
1184 & 19897647296712343741 & 183487771 & 465706 \\
1187 & 20253863242769197141 & 184888441 & 468073 \\
1190 & 20615531373101560501 & 186296221 & 470446 \\
1193 & 20982721052901472525 & 187711129 & 472825 \\
1196 & 21355502351568381865 & 189133183 & 475210 \\
1199 & 21733946048057518801 & 190562401 & 477601 \\
1202 & 22118123636255278801 & 191998801 & 479998 \\
1205 & 22508107330381686001 & 193442401 & 482401 \\
1208 & 22903970070420004645 & 194893219 & 484810 \\
1211 & 23305785527573566525 & 196351273 & 487225 \\
1214 & 23713628109749882461 & 197816581 & 489646 \\
1217 & 24127572967072105861 & 199289161 & 492073 \\
1220 & 24547695997417916401 & 200769031 & 494506 \\
1223 & 24974073851985891865 & 202256209 & 496945 \\
1226 & 25406783940889436185 & 203750713 & 499390 \\
1229 & 25845904438778331721 & 205252561 & 501841 \\
1232 & 26291514290487983821 & 206761771 & 504298 \\
1235 & 26743693216716425701 & 208278361 & 506761 \\
1238 & 27202521719729151685 & 209802349 & 509230 \\
1241 & 27668081089091846845 & 211333753 & 511705 \\
1244 & 28140453407431081081 & 212872591 & 514186 \\
1247 & 28619721556223035681 & 214418881 & 516673 \\
1250 & 29105969221610330401 & 215972641 & 519166 \\
1253 & 29599280900247019105 & 217533889 & 521665 \\
1256 & 30099741905171822005 & 219102643 & 524170 \\
1259 & 30607438371709662541 & 220678921 & 526681 \\
1262 & 31122457263401576941 & 222262741 & 529198 \\
1265 & 31644886377963064501 & 223854121 & 531721 \\
1268 & 32174814353270946625 & 225453079 & 534250 \\
1271 & 32712330673378802665 & 227059633 & 536785 \\
1274 & 33257525674561050601 & 228673801 & 539326 \\
1277 & 33810490551385740601 & 230295601 & 541873 \\
1280 & 34371317362816129501 & 231925051 & 544426 \\
1283 & 34940099038341104245 & 233562169 & 546985 \\
1286 & 35516929384134522325 & 235206973 & 549550 \\
1289 & 36101903089243537261 & 236859481 & 552121 \\
1292 & 36695115731805977161 & 238519711 & 554698 \\
1295 & 37296663785296844401 & 240187681 & 557281 \\
1298 & 37906644624804004465 & 241863409 & 559870 \\
1301 & 38525156533333131985 & 243546913 & 562465 \\
1304 & 39152298708141982021 & 245238211 & 565066 \\
1307 & 39788171267104054621 & 246937321 & 567673 \\
1310 & 40432875255101720701 & 248644261 & 570286 \\
1313 & 41086512650448877285 & 250359049 & 572905 \\
1316 & 41749186371343200145 & 252081703 & 575530 \\
1319 & 42421000282348061881 & 253812241 & 578161 \\
1322 & 43102059200904183481 & 255550681 & 580798 \\
1325 & 43792468903871087401 & 257297041 & 583441 \\
1328 & 44492336134098420205 & 259051339 & 586090 \\
1331 & 45201768607027212805 & 260813593 & 588745 \\
1334 & 45920875017321146341 & 262583821 & 591406 \\
1337 & 46649765045527891741 & 264362041 & 594073 \\
1340 & 47388549364770591001 & 266148271 & 596746 \\
1343 & 48137339647469548225 & 267942529 & 599425 \\
1346 & 48896248572094198465 & 269744833 & 602110 \\
1349 & 49665389829945422401 & 271555201 & 604801 \\
1352 & 50444878131968274901 & 273373651 & 607498 \\
1355 & 51234829215595195501 & 275200201 & 610201 \\
1358 & 52035359851619768845 & 277034869 & 612910 \\
1361 & 52846587851101103125 & 278877673 & 615625 \\
1364 & 53668632072298894561 & 280728631 & 618346 \\
1367 & 54501612427639245961 & 282587761 & 621073 \\
1370 & 55345649890711307401 & 284455081 & 623806 \\
1373 & 56200866503294807065 & 286330609 & 626545 \\
1376 & 57067385382418540285 & 288214363 & 629290 \\
1379 & 57945330727449884821 & 290106361 & 632041 \\
1382 & 58834827827215410421 & 292006621 & 634798 \\
1385 & 59736003067152650701 & 293915161 & 637561 \\
1388 & 60648983936493105385 & 295831999 & 640330 \\
1391 & 61573899035476540945 & 297757153 & 643105 \\
1394 & 62510878082596657681 & 299690641 & 645886 \\
1397 & 63460051921878191281 & 301632481 & 648673 \\
1400 & 64421552530185516901 & 303582691 & 651466 \\
1403 & 65395513024562823805 & 305541289 & 654265 \\
1406 & 66382067669605928605 & 307508293 & 657070 \\
1409 & 67381351884865795141 & 309483721 & 659881 \\
1412 & 68393502252283829041 & 311467591 & 662698 \\
1415 & 69418656523659015001 & 313459921 & 665521 \\
1418 & 70456953628146964825 & 315460729 & 668350 \\
1421 & 71508533679790944265 & 317470033 & 671185 \\
1424 & 72573537985084946701 & 319487851 & 674026 \\
1427 & 73652109050568881701 & 321514201 & 676873 \\
1430 & 74744390590455946501 & 323549101 & 679726 \\
1433 & 75850527534292248445 & 325592569 & 682585 \\
1436 & 76970666034648746425 & 327644623 & 685450 \\
1439 & 78104953474845579361 & 329705281 & 688321 \\
1442 & 79253538476708849761 & 331774561 & 691198 \\
1445 & 80416570908359930401 & 333852481 & 694081 \\
1448 & 81594201892037362165 & 335939059 & 696970 \\
1451 & 82786583811951411085 & 338034313 & 699865 \\
1454 & 83993870322171352621 & 340138261 & 702766 \\
1457 & 85216216354545551221 & 342250921 & 705673 \\
1460 & 86453778126654403201 & 344372311 & 708586 \\
1463 & 87706713149796210985 & 346502449 & 711505 \\
1466 & 88975180237006056745 & 348641353 & 714430 \\
1469 & 90259339511107743481 & 350789041 & 717361 \\
1472 & 91559352412798871581 & 352945531 & 720298 \\
1475 & 92875381708769118901 & 355110841 & 723241 \\
1478 & 94207591499851792405 & 357284989 & 726190 \\
1481 & 95556147229208719405 & 359467993 & 729145 \\
1484 & 96921215690548546441 & 361659871 & 732106 \\
1487 & 98302965036378513841 & 363860641 & 735073 \\
1490 & 99701564786289774001 & 366070321 & 738046 \\
2378 & 432684460 & 604 & 103 \\
2386 & 29437553530 & 4970 & 420 \\
4980 & 24264913354964425 & 3122317 & 30810 \\
9325 & 3176083959788026 & 825436 & 12691 \\
9525 & 16195753597485 & 58322 & 2169 \\
16420 & 913053565546276 & 333506 & 6936 \\
19605 & 5519583702676 & 23731 & 1191 \\
31265 & 90525801730 & 2407 & 259 \\
31368 & 17147031694579605 & 1045635 & 14858 \\
83135 & 31148407558500 & 27375 & 1310 \\
125070 & 890348736143873526 & 3773306 & 34956 \\
210903 & 10290361955160 & 9879 & 664 \\
223613 & 421687634347915 & 61414 & 2245 \\

\bottomrule
\caption{Polygonal Cannonball Numbers}
\label{tab_ugly}
\end{longtable}

\end{document}
